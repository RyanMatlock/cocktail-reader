\begin{Cocktail}{Alexander}
	\begin{Ingredients}
	\item \SI{1}{\oz} gin
	\item \SI{1}{\oz} cr\`{e}me de cacao
	\item \SI{1}{\oz} cream
	\end{Ingredients}
	
	\begin{Instructions}
	Shake ingredients with ice, strain into a coupe, and sprinkle with grated nutmeg.
	
	A \InlineCocktail{Brandy Alexander} substitutes brandy for gin, and a \InlineCocktail{Coffee Alexander} substitutes coffee liqueur for gin.
	\end{Instructions}
\end{Cocktail}

\begin{Cocktail*}{Almond Drop} # a comment on the Almond drop
	\begin{Ingredients}
	\item \SI{1}{\oz} vodka
	\item \SI{1}{\oz} amaretto
	\item \SI{\half}{\oz} simple syrup
	\item \SI{1}{\oz} Meyer lemon juice
	\end{Ingredients}
	
	\begin{Instructions}
	Shake ingredients with ice, strain into a coupe, and garnish with a maraschino cherry.
	\end{Instructions}
\end{Cocktail*}

\begin{Cocktail}{Aviation}
	\begin{Ingredients}
	\item \SI{2}{\oz} gin
	\item \SI{\half}{\oz} maraschino liqueur
	\item \SI{\half}{\oz} lemon juice
	\item \SI{\quarter}{\oz} cr\`{e}me de violette
	\end{Ingredients}
	
	\begin{Instructions}
	Shake ingredients with ice, strain into a coupe, and garnish with a maraschino cherry.
	\end{Instructions}
\end{Cocktail}

\begin{Cocktail}{Corpse Reviver \No 2}
	\begin{Ingredients}
	\item \SI{\threequarters}{\oz} gin
	\item \SI{\threequarters}{\oz} triple sec
	\item \SI{\threequarters}{\oz} \Lillet
	\item \SI{\threequarters}{\oz} lemon juice
	\item a \si{\dash} absinthe
	\end{Ingredients}
	
	\begin{Instructions}\vspace{-3ex}
	Shake ingredients with ice, strain into a coupe, and garnish with a lemon twist or a maraschino cherry.
	
	To make a \InlineCocktail{Kentucky Corpse Reviver}, substitute bourbon for gin, and garnish with a sprig of mint.
	\end{Instructions}
\end{Cocktail}



\begin{Cocktail}{Sidecar}
	\begin{Ingredients}
	\item \SI{2}{\oz} brandy
	\item \SI{\half}{\oz} triple sec
	\item \SI{\quarter}{\oz} lemon juice
	\end{Ingredients}
	
	\begin{Instructions}
	Shake the ingredients with ice, strain into a sugar-rimmed coupe, and garnish with a lemon twist.
	\end{Instructions}
\end{Cocktail}

\begin{Cocktail*}{It Happened One Night}
	\begin{Ingredients}\normalsize
	\item \SI{1\half*}{\oz} tequila
	\item \SI{\half}{\oz} white rum
	\item \SI{\half}{\oz} amaretto
	\item \SI{\half}{\oz} triple sec
	\item \SI{\half}{\oz} dry vermouth
	\item a \si{\dash} Peychaud's bitters
	\item \SIrange{4}{6}{\oz} tonic water
	\end{Ingredients}
	
	\begin{Instructions}
	Pour ingredients into an ice-filled highball glass, stir, and garnish with a maraschino cherry.
	\end{Instructions}
\end{Cocktail*}

\begin{Cocktail}{Ch\^atelaine}
	\begin{Ingredients}
	\item \SI{1\half*}{\oz} crisp white wine
	\item \SI{\threequarters}{\oz} gin or vodka
	\item \SI{2}{\tsp} St.\ Germain
	\item \SI{\threequarters}{\oz} pomegranate juice
	\end{Ingredients}
	
	\begin{Instructions}
	Shake ingredients with ice, strain into a coupe, and garnish with pomegranate arils.
	\end{Instructions}
\end{Cocktail}

\begin{Cocktail*}{The Pear Dream} %% (old name was Mental Gloss; current name comes from a Kids in the Hall sketch)
	\begin{Ingredients}
	\item \SI{1\half*}{\oz} brandy%\footnote{Korbel VSOP used originally.}
	\item \SI{\half}{\oz} St.\ Germain
	\item \SI{\half}{\oz} pear juice
	\item \SI{\quarter}{\oz} lemon juice
	\item \SI{\quarter}{\oz} Laird's straight apple brandy
	\end{Ingredients}
	
	\begin{Instructions}
	Shake ingredients with ice, strain into a coupe, and garnish with a slice of fresh pear.
	\end{Instructions}
\end{Cocktail*}

\begin{Cocktail*}{Julie's Drink} % named for Matt's sister, Juliet.  Also, you owe Monica a drink named after her that has bourbon, Chartreuse, and Chambourd (plus other stuff)
	\begin{Ingredients}
	\item \SI{1\half*}{\oz} bourbon
	\item \SI{\half}{\oz} St.\ Germain
	\item \SI{\half}{\oz} maraschino liqueur%\footnote{Maraska maraschino liqueur used originally.}
	\item \SI{\quarter}{\oz} Campari
	\end{Ingredients}
	
	\begin{Instructions}
	Stir ingredients with ice%\footnote{The first drink was accidentally shaken, but it was still a hit.}
	, strain into a coupe, and garnish with a lemon twist.
	\end{Instructions}
\end{Cocktail*}
%% end Sarah's fancy cocktail party notes

\begin{Cocktail*}{Mood Ring} % came up with this one night while Laura was over -- Sarah came up with the name
%% also, consider trying this with dry vermouth
	\begin{Ingredients}
	\item \SI{1\half*}{\oz} Highland or Speyside scotch
	\item \SI{\half}{\oz} St.\ Germain
	\item \SI{\half}{\oz} cr\`eme de violette
	\item \SI{\half}{\oz} sweet vermouth
	\item a couple \si{\dashes} Peychaud's bitters
	\end{Ingredients}
	
	\begin{Instructions}\small
	Stir all ingredients except cr\`eme de violette with ice, strain into an ice-filled old fashioned glass, and pour in cr\`eme de violette over the back of a cocktail spoon.  The latter will settle to the bottom, and over time, the drink will change color, almost like a mood ring.
	\end{Instructions}
\end{Cocktail*}

%% from LA Times, May 4, 2013
\begin{Cocktail}{American Poet}
	\begin{Ingredients}
	\item \SI{2}{\oz} bourbon\footnote{Buffalo Trace is preferred.}
	\item \SI{\half}{\oz} green Chartreuse
	\item \SI{\half}{\oz} maraschino liqueur
	\item \SI{\half}{\oz} lime juice
	\item a couple \si{\dashes} orange bitters
	\end{Ingredients}
	
	\begin{Instructions}
	Shake ingredients with ice, strain into a coupe, and garnish with an orange twist.
	\end{Instructions}
\end{Cocktail}

\begin{EOCocktail}{Grand Fashioned}
	\begin{Ingredients}
	\item \SI{2}{\oz} Grand Marnier
	\item \SI{\threequarters}{\oz} lime juice
	\item 3 \si{\dashes} Angostura bitters
	\item \SI{1}{\tsp} superfine sugar
	\item 3 blood orange wedges, peeled
	\end{Ingredients}
	
	\begin{Instructions}
	Muddle the sugar, bitters, and oranges, then add the Grand Marnier and juice.  Add enough ice cubes so that the whole mixture---including the soon-to-be-used ice---fits in an old fashioned glass.  Shake hard but briefly and pour the whole thing, \emph{unstrained}, into an old fashioned glass.
	\end{Instructions}
\end{EOCocktail}

%% you'll have to actually try this
\begin{Cocktail*}{Martini Vanilli}
	\begin{Ingredients}
	\item \SI{2}{\oz} oak-aged Meyer lemon New Amsterdam gin
	\item \SI{\half}{\oz} dry vermouth
	\item \SI{\half}{\oz} sweet vermouth
	\item a \si{\dash} Angostura bitters %% ?????
	\end{Ingredients}
	
	\begin{Instructions}
	Stir ingredients with ice, strain into a coupe, and garnish with a lemon twist.
	\end{Instructions}
\end{Cocktail*}

%%% PDT logo
\newcommand\PDTLogo{\textit{\raisebox{1ex}{P}\hspace{-0.27em}D\hspace{-0.17em}\raisebox{-1ex}{T}}}
\newenvironment{PDTCocktail}[1]
	{%
		\begin{Cocktail}[\Attribution{\PDTLogo}]{#1}
	}
	{%
		\end{Cocktail}
	}
%% starred version indicates that cocktail was taken from PDT cocktail book but is *not* a PDT original
\newcommand\PDTStarLogo{\textit{\raisebox{1ex}{P}\hspace{-0.27em}D\hspace{-0.17em}\raisebox{-1ex}{T}\,\raisebox{1ex}[0pt][0pt]{$^*$}}}
\newenvironment{PDTCocktail*}[1]
	{%
		\begin{Cocktail}[\Attribution{\PDTStarLogo}]{#1}
	}
	{%
		\end{Cocktail}
	}
\begin{PDTCocktail*}{Aviation} %%% make sure to specify gin family later
	\begin{Ingredients}
	\item \SI{2}{\oz} London dry gin
	\item \SI{\threequarters}{\oz} lemon juice
	\item \SI{\half}{\oz} Luxardo maraschino liqueur
	\item \SI{\quarter}{\oz} cr\`eme de violette
	\end{Ingredients}
	
	\begin{Instructions}
	Shake ingredients with ice, and strain into a coupe.
	\end{Instructions}
\end{PDTCocktail*}


%% John Squier logo
\newcommand\JohnSquierLogo{\textit{\raisebox{1ex}{J}\hspace{-0.22em}S}}
\newenvironment{JSCocktail}[1]
	{%
		\begin{Cocktail}[\Attribution{\JohnSquierLogo}]{#1}
	}
	{%
		\end{Cocktail}
	}
\begin{JSCocktail}{Bobby Burns}
	\begin{Ingredients}
	\item \SI{2}{\oz} scotch %% Bowmore Legend was used
	\item \SI{1}{\oz} sweet vermouth
	\item \SI{\third}{\oz} B\'en\'edictine
	\end{Ingredients}
\end{JSCocktail}

\begin{Cocktail}[\Attribution{DdG}]{Bobby Burns}
	\begin{Ingredients}
	\item \SI{2}{\oz} Highland scotch
	\item \SI{\threequarters}{\oz} sweet vermouth
	\item \SI{\half}{\oz} B\'en\'edictine
	\end{Ingredients}
\end{Cocktail}

\begin{Cocktail*}{Shylock} % rye whiskey -> rye bread -> Jewish food / Fernet -> Italians / Jewish Italians -> Shylock from The Merchant of Venice
	\begin{Ingredients}
	\item \SI{1\half*}{\oz} Rittenhouse rye whiskey
	\item \SI{\half}{\oz} Fernet Branca
	\item \SI{\half}{\oz} grenadine
	\item \SI{\half}{\oz} lemon juice
	\item \SI{\quarter}{\oz} lemon juice
	\end{Ingredients}
	
	\begin{Instructions}
	Shake ingredients with ice, strain into a coupe, and garnish with a lemon twist.
	\end{Instructions}
	\FantabulonAvalanche
\end{Cocktail*}

\begin{Cocktail*}{Northern Lights}
	\begin{Ingredients}
	\item \SI{1}{\oz} keffir lime leaf and lavender New Amsterdam gin
	\item \SI{1}{\oz} green Chartreuse
	\item \SI{1}{\oz} lime juice
	\item \SIrange{1}{2}{\oz} champagne
	\end{Ingredients}
	
	\begin{Instructions}
	Shake gin, Chartreuse, and lime juice with ice, strain into an ice-filled Collins glass, top with champagne, and garnish with a lime wheel.
	\end{Instructions}
	\FantabulonAvalanche
\end{Cocktail*}

%% AltCocktail
\newcounter{AltCocktail}
\setcounter{AltCocktail}{1}
\newcommand\FNSymbolLogo[1][1]{\setcounter{AltCocktail}{#1}{\Large{\fnsymbol{AltCocktail}}}\setcounter{AltCocktail}{1}}
\newenvironment{AltCocktail}[2][1]
	{%
		\begin{Cocktail}[\Attribution{\FNSymbolLogo[#1]}]{#2}
	}
	{%
		\end{Cocktail}
	}
%% source: http://whiteonricecouple.com/recipes/margarita-recipe/
\begin{AltCocktail}{Margarita}
	\begin{Ingredients}
	\item \SI{1\half*}{\oz} reposado tequila
	\item \SI{1\half*}{\oz} lime juice
	\item \SI{1}{\oz} simple syrup
	\item 3 \si{\dashes} orange bitters
	\end{Ingredients}
	
	\begin{Instructions}
	Salt the rim of an old fashioned glass and add one or two large ice cubes.  Shake the ingredients with ice, and strain into the prepared glass.
	\end{Instructions}
\end{AltCocktail}

%% Comme Ca
\newcommand\CommeCaLogo{\textit{\LARGE\raisebox{0.45ex}{c}\hspace{-0.27em}\c c}}
\newenvironment{CCCocktail}[1]
	{%
		\begin{Cocktail}[\Attribution{\CommeCaLogo}]{#1}
	}
	{%
		\end{Cocktail}
	}
\begin{CCCocktail}{Ramble}
	\begin{Ingredients}
	\item gin
	\item lemon juice
	\item sugar
	\item raspberries
	\end{Ingredients}
	
	\begin{Instructions}
	served on the rocks in an old fashioned glass
	\end{Instructions}
\end{CCCocktail}

\begin{CCCocktail}{Don Guillermo}
	\begin{Ingredients}
	\item Peligroso tequila reposado % the reposado is a guess, the brand comes from it specifying 84 proof
	\item lime juice
	\item brown sugar
	\item green \vep\ Chartreuse
	\item cr\`eme de cassis
	\end{Ingredients}
	
	\begin{Instructions}
	served on the rocks in an old fashioned glass (Chartreuse sinks to the bottom? or maybe the cr\`eme de cassis?)
	\end{Instructions}
\end{CCCocktail}

\begin{PDTCocktail}{Apricot Flip}
	\begin{Ingredients}
	\item \SI{2}{\oz} Hine \vsop\ cognac
	\item \SI{\threequarters}{\oz} Rothman \&\ Winter Orchard Apricot
	\item \SI{\half}{\oz} simple syrup
	\item 1 egg white
	\end{Ingredients}
	
	\begin{Instructions}
	Dry shake ingredients, then shake with ice, strain into an old fashioned glass, and garnish with grated nutmeg.
	\end{Instructions}
\end{PDTCocktail}